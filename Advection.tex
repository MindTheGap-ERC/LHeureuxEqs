\documentclass[11pt, letterpaper]{article}
\usepackage{amsmath}
\title{Advection}
\date{22 December 2023}

\begin{document}
\maketitle

In the article:
\begin{equation}
    U = 1 - \frac{K(\phi^0)}{S}(1-\phi^0)\big(\frac{\rho_s^0}{\rho_w} - 1 \big) + \frac{K(\phi)}{S}(1-\phi) \big(\frac{\rho_s}{\rho_w}-1\big)
\end{equation}

\begin{equation}
    W = 1 - \frac{K(\phi^0)}{S}(1-\phi^0) \big(\frac{\rho_s^0}{\rho_w}-1) - \frac{K(\phi)}{S} \frac{(1-\phi)^2}{\phi}\big(\frac{\rho_s}{\rho_w}-1\big)
\end{equation}
Just to re-calculate how these expressions are evaluated in Python, I reversed the code into equations.
In \verb|evolution_rate|:
\begin{verbatim}
    U = self.presum + self.rhorat * Phi ** 3 * F /one_minus_Phi
\end{verbatim}
where:
\begin{verbatim}
    self.presum = 1 - self.rhorat0 * self.Phi0 ** 3 * (1 - np.exp(10 - 10 / self.Phi0)) / (1 - self.Phi0)     
\end{verbatim}

I need to rewrite it back into equations to be able to read it:

\begin{equation}
    rhorat_0 = (\frac{\rho_s^0}{\rho_w} - 1) \frac{\beta}{S}
\end{equation}
Substituting this into \verb|presum|:
\begin{equation}
    presum = 1 - (\frac{\rho_s^0}{\rho_w} - 1) \frac{\beta}{S} \times \frac{\phi_0^3}{(1 - \phi_0)} \left(1 - \exp\left(10 - \frac{10}{\phi_0} \right) \right)  
\end{equation}
\begin{verbatim}
    self.rhorat = (self.rhos / self.rhow - 1) * self.beta / self.sedimentationrate
\end{verbatim}
Which is:
\begin{equation}
    rhorat = (\frac{\rho_s}{\rho_w} - 1)\frac{\beta}{S}
\end{equation}
So $U$ from the code is calculated as follows:
\begin{equation}
    U = 1 - (\frac{\rho_s^0}{\rho_w} - 1) \frac{\beta}{S} \times \frac{\phi_0^3}{(1 - \phi_0)} \left(1 - \exp\left(10 - \frac{10}{\phi_0} \right) \right) + (\frac{\rho_s}{\rho_w} - 1)\frac{\beta}{S} \phi^3\times\frac{F}{1-\phi}
\end{equation}
\begin{equation}
    U = 1 - \frac{K(\phi^0)}{S}(1-\phi^0)\big(\frac{\rho_s^0}{\rho_w} - 1 \big) + \frac{K(\phi)}{S}(1-\phi) \big(\frac{\rho_s}{\rho_w}-1\big)
\end{equation}

\end{document}