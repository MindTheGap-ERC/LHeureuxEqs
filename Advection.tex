\documentclass[11pt, letterpaper]{article}
\usepackage{amsmath}
\title{The diagenetic model by L'Heureux (2018)}
\author{Emilia Jarochowska & Niklas Hohmann}
\date{22 December 2023}

\begin{document}
\maketitle

In the article:
\begin{equation}
    U = 1 - \frac{K(\phi^0)}{S}(1-\phi^0)\big(\frac{\rho_s^0}{\rho_w} - 1 \big) + \frac{K(\phi)}{S}(1-\phi) \big(\frac{\rho_s}{\rho_w}-1\big)
\end{equation}

\begin{equation}
    W = 1 - \frac{K(\phi^0)}{S}(1-\phi^0) \big(\frac{\rho_s^0}{\rho_w}-1) - \frac{K(\phi)}{S} \frac{(1-\phi)^2}{\phi}\big(\frac{\rho_s}{\rho_w}-1\big)
\end{equation}

Just to re-calculate how these expressions are evaluated in Python, I turned the code into equations

\begin{equation}
    rhorat_0 = \frac{\rho_s^0}{rho_w - 1} \frac{\beta}{S}
\end{equation}

\begin{equation}
    presum = 1 - \frac{\rho_s^0}{rho_w - 1} \frac{\beta}{S} \times \frac{\phi_0^3}{(1 - \phi_0)} \left(1 - \exp\left(10 - \frac{10}{\phi_0} \right) \right)  
\end{equation}

Two questions:
The second part of this product, $\frac{\phi_0^3}{(1 - \phi_0)} \left(1 - \exp\left(10 - \frac{10}{\phi_0} \right) \right)$, is $K$ from Eq. 15. But in Eq the denominator is squared $(1 - \phi_0)^2$, but here it is not.
Secondly, I don't know why presum uses always $\phi_0$. Advection shouldn't really change if we don't make it dependent on the changes in $\phi$?
\end{document}